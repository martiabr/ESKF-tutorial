%%%%%%%%%%%%%%%%%%%%%%%%%%%%%%%%%%%%%%%%%
% Beamer Presentation
% LaTeX Template
% Version 1.0 (10/11/12)
%
% This template has been downloaded from:
% http://www.LaTeXTemplates.com
%
% License:
% CC BY-NC-SA 3.0 (http://creativecommons.org/licenses/by-nc-sa/3.0/)
%
%%%%%%%%%%%%%%%%%%%%%%%%%%%%%%%%%%%%%%%%%

%----------------------------------------------------------------------------------------
%	PACKAGES AND THEMES
%----------------------------------------------------------------------------------------

\documentclass{beamer}

\mode<presentation> {

% The Beamer class comes with a number of default slide themes
% which change the colors and layouts of slides. Below this is a list
% of all the themes, uncomment each in turn to see what they look like.

%\usetheme{default}
%\usetheme{AnnArbor}
%\usetheme{Antibes}
%\usetheme{Bergen}
%\usetheme{Berkeley}
%\usetheme{Berlin}
%\usetheme{Boadilla}
%\usetheme{CambridgeUS}
%\usetheme{Copenhagen}
%\usetheme{Darmstadt}
%\usetheme{Dresden}
%\usetheme{Frankfurt}
%\usetheme{Goettingen}
%\usetheme{Hannover}
%\usetheme{Ilmenau}
%\usetheme{JuanLesPins}
%\usetheme{Luebeck}
\usetheme{Madrid}
%\usetheme{Malmoe}
%\usetheme{Marburg}
%\usetheme{Montpellier}
%\usetheme{PaloAlto}
%\usetheme{Pittsburgh}
%\usetheme{Rochester}
%\usetheme{Singapore}
%\usetheme{Szeged}
%\usetheme{Warsaw}

% As well as themes, the Beamer class has a number of color themes
% for any slide theme. Uncomment each of these in turn to see how it
% changes the colors of your current slide theme.

%\usecolortheme{albatross}
%\usecolortheme{beaver}
%\usecolortheme{beetle}
%\usecolortheme{crane}
%\usecolortheme{dolphin}
%\usecolortheme{dove}
%\usecolortheme{fly}
%\usecolortheme{lily}
%\usecolortheme{orchid}
%\usecolortheme{rose}
%\usecolortheme{seagull}
%\usecolortheme{seahorse}
%\usecolortheme{whale}
%\usecolortheme{wolverine}

%\setbeamertemplate{footline} % To remove the footer line in all slides uncomment this line
%\setbeamertemplate{footline}[page number] % To replace the footer line in all slides with a simple slide count uncomment this line

\setbeamertemplate{navigation symbols}{} % To remove the navigation symbols from the bottom of all slides uncomment this line
}

\usepackage{graphicx} % Allows including images
\usepackage{booktabs} % Allows the use of \toprule, \midrule and \bottomrule in tables

%----------------------------------------------------------------------------------------
%	TITLE PAGE
%----------------------------------------------------------------------------------------

\title[ESKF Introduction]{Introduction to the Error-state Kalman filter} % The short title appears at the bottom of every slide, the full title is only on the title page

\author{Martin Brandt} % Your name
\date{\today} % Date, can be changed to a custom date

\begin{document}

\begin{frame}
\titlepage % Print the title page as the first slide
\end{frame}

\begin{frame}
\frametitle{Overview} % Table of contents slide, comment this block out to remove it
\tableofcontents % Throughout your presentation, if you choose to use \section{} and \subsection{} commands, these will automatically be printed on this slide as an overview of your presentation
\end{frame}

%----------------------------------------------------------------------------------------
%	PRESENTATION SLIDES
%----------------------------------------------------------------------------------------
\section{Motivation}

\begin{frame}
    \frametitle{Motivation}
    \begin{figure}
    \includegraphics[width=0.8\linewidth]{adcs.png}
    \end{figure}
    The controller needs to know the attitude in order to control it, but there is no way to measure it directly $\rightarrow$ we have to estimate it!
\end{frame}

%------------------------------------------------

\section{State space models}

\begin{frame}
    \frametitle{State space models}
    Want to represent an arbitrary system of differential equations in vector form. In general: $\dot{x} = f(x, u)$

    \begin{block}{Continous LTI state space model}
        \begin{equation}
        \begin{aligned}
            \dot{x} &= A x + B u \\
            y &=C x + D u
        \end{aligned}
        \end{equation}
    \end{block}

    \begin{block}{Discrete LTI state space model}
        \begin{equation}
        \begin{aligned}
            x[k+1] &= A x[k] + B u[k] \\
            y[k] &= C x[k] + D u[k]
        \end{aligned}
        \end{equation}
    \end{block}
\end{frame}

%------------------------------------------------

\begin{frame}
    \frametitle{Mass-spring-damper example}

    \begin{block}{How you are used to seeing it}
        \begin{equation}
            m \ddot{x}+d \dot{x} + ky=u
        \end{equation}
    \end{block}

    \begin{block}{State space representation}
        \begin{equation}
        \left[\begin{array}{c}{{\dot{x_1}}} \\ {\dot{x_2}}\end{array}\right] =\left[\begin{array}{cc}{0} & {1} \\ {-\frac{k}{m}} & {-\frac{d}{m}}\end{array}\right] \left[\begin{array}{c}{{x_1}} \\ {x_2}\end{array}\right] +\left[\begin{array}{c}{0} \\ {\frac{1}{m}}\end{array}\right] u
        \end{equation}
    \end{block}
\end{frame}

%------------------------------------------------

\section{The Kalman filter}

\begin{frame}
    \frametitle{Luenberger observer}
    Let's say we have a state space model of our system. How would we try to estimate the states of the system? 

    \begin{block}{The logical first try (open-loop observer)}
        \begin{equation}
            \dot{\hat{x}} = A \hat{x} + B u 
        \end{equation}
    \end{block}

    But because of modeling uncertainty our estimate will quickly diverge from the real value $\rightarrow$ include a correction term based on measurements (closing the loop) $\rightarrow$ Luenberger observer

    \begin{block}{The Luenberger observer}
        \begin{equation}
            \dot{\hat{x}} = A \hat{x} + B u + L (y - \hat{y}), \quad \hat{y} = C \hat{x}
        \end{equation}
    \end{block}
    But how do we decide the gain $L$?
\end{frame}

%------------------------------------------------

\begin{frame}
    \frametitle{The Kalman filter}
    Let us first assume that our process model and measurement model includes \textbf{normally distributed} noise:
    \begin{block}{Stochastic LTI system}
        \begin{equation}
            \begin{aligned}
                \dot{x} &= A x + B u + w \\
                y &=C x + D u + v
            \end{aligned}
        \end{equation}
    \end{block}
    The Kalman Filter is the optimal Luenberger observer for this system, in the sense that it minimizes \textbf{mean squared error}, i.e. $E\{(x-\hat{x})^2\}$. 
\end{frame}

%------------------------------------------------

\begin{frame}
    \frametitle{The Kalman filter equations}
    For the discrete case (which is what you would implement on a microcontroller) the Kalman filter equations are:
    \begin{figure}
        \includegraphics[width=0.8\linewidth]{kf_equations.png}
    \end{figure}
    The details are not important here, but note the predict + correct steps. 
\end{frame}

%------------------------------------------------

\begin{frame}
    \frametitle{The satellite kinematics and kinetics}
    Let's try to apply the Kalman filter to our satellite:
    \begin{block}{Satellite kinematics}
        \begin{equation}
            \begin{aligned}
                \dot{\mathbf{q}}=\frac{1}{2} \mathbf{q} \otimes \boldsymbol{\omega}
            \end{aligned}
        \end{equation}
    \end{block}
    \begin{block}{Satellite kinetics}
        \begin{equation}
            \begin{aligned}
                \dot{\omega}=J^{-1}\left[\mathbf{L}-\omega \times\left(J \omega\right)\right]
            \end{aligned}
        \end{equation}
    \end{block}
    Problem: the system is highly nonlinear, so the Kalman filter cannot be directly applied (since it assumes a linear model).
\end{frame}

%------------------------------------------------

\begin{frame}
    \frametitle{Extended Kalman filter}
    The easiest solution to this problem would be to linearize the nonlinear dynamics at each timestep $\rightarrow$ Extended Kalman filter (EKF). \\~\\

    Problem: modeling uncertainty - the kinetics require that we know the inertia matrix of the satellite and since the dynamics are highly nonlinear the EKF might diverge:( \\~\\

    $\rightarrow$ Drop the kinetics, only use the kinematics: $\dot{\mathbf{q}}=\frac{1}{2} \mathbf{q} \otimes \boldsymbol{\omega}$. \\
    We let $\omega$ be the "control input", which we measure with the IMU. \\~\\
\end{frame}

\begin{frame}
    \frametitle{Error-state Kalman filter}
    Now we are getting closer to a somewhat usage algorithm, but the dynamics are still highly nonlinear, which means the EKF will behave poorly.
\end{frame}

%------------------------------------------------

\section{The Error-state Kalman filter}

%------------------------------------------------
\section{First Section} % Sections can be created in order to organize your presentation into discrete blocks, all sections and subsections are automatically printed in the table of contents as an overview of the talk
%------------------------------------------------

\subsection{Subsection Example} % A subsection can be created just before a set of slides with a common theme to further break down your presentation into chunks

\begin{frame}
\frametitle{Paragraphs of Text}
Sed iaculis dapibus gravida. Morbi sed tortor erat, nec interdum arcu. Sed id lorem lectus. Quisque viverra augue id sem ornare non aliquam nibh tristique. Aenean in ligula nisl. Nulla sed tellus ipsum. Donec vestibulum ligula non lorem vulputate fermentum accumsan neque mollis.\\~\\

Sed diam enim, sagittis nec condimentum sit amet, ullamcorper sit amet libero. Aliquam vel dui orci, a porta odio. Nullam id suscipit ipsum. Aenean lobortis commodo sem, ut commodo leo gravida vitae. Pellentesque vehicula ante iaculis arcu pretium rutrum eget sit amet purus. Integer ornare nulla quis neque ultrices lobortis. Vestibulum ultrices tincidunt libero, quis commodo erat ullamcorper id.
\end{frame}

%------------------------------------------------

\begin{frame}
\frametitle{Bullet Points}
\begin{itemize}
\item Lorem ipsum dolor sit amet, consectetur adipiscing elit
\item Aliquam blandit faucibus nisi, sit amet dapibus enim tempus eu
\item Nulla commodo, erat quis gravida posuere, elit lacus lobortis est, quis porttitor odio mauris at libero
\item Nam cursus est eget velit posuere pellentesque
\item Vestibulum faucibus velit a augue condimentum quis convallis nulla gravida
\end{itemize}
\end{frame}

%------------------------------------------------

\begin{frame}
\frametitle{Blocks of Highlighted Text}
\begin{block}{Block 1}
Lorem ipsum dolor sit amet, consectetur adipiscing elit. Integer lectus nisl, ultricies in feugiat rutrum, porttitor sit amet augue. Aliquam ut tortor mauris. Sed volutpat ante purus, quis accumsan dolor.
\end{block}

\begin{block}{Block 2}
Pellentesque sed tellus purus. Class aptent taciti sociosqu ad litora torquent per conubia nostra, per inceptos himenaeos. Vestibulum quis magna at risus dictum tempor eu vitae velit.
\end{block}

\begin{block}{Block 3}
Suspendisse tincidunt sagittis gravida. Curabitur condimentum, enim sed venenatis rutrum, ipsum neque consectetur orci, sed blandit justo nisi ac lacus.
\end{block}
\end{frame}

%------------------------------------------------

\begin{frame}
\frametitle{Multiple Columns}
\begin{columns}[c] % The "c" option specifies centered vertical alignment while the "t" option is used for top vertical alignment

\column{.45\textwidth} % Left column and width
\textbf{Heading}
\begin{enumerate}
\item Statement
\item Explanation
\item Example
\end{enumerate}

\column{.5\textwidth} % Right column and width
Lorem ipsum dolor sit amet, consectetur adipiscing elit. Integer lectus nisl, ultricies in feugiat rutrum, porttitor sit amet augue. Aliquam ut tortor mauris. Sed volutpat ante purus, quis accumsan dolor.

\end{columns}
\end{frame}

%------------------------------------------------
\section{Second Section}
%------------------------------------------------

\begin{frame}
\frametitle{Table}
\begin{table}
\begin{tabular}{l l l}
\toprule
\textbf{Treatments} & \textbf{Response 1} & \textbf{Response 2}\\
\midrule
Treatment 1 & 0.0003262 & 0.562 \\
Treatment 2 & 0.0015681 & 0.910 \\
Treatment 3 & 0.0009271 & 0.296 \\
\bottomrule
\end{tabular}
\caption{Table caption}
\end{table}
\end{frame}

%------------------------------------------------

\begin{frame}
\frametitle{Theorem}
\begin{theorem}[Mass--energy equivalence]
$E = mc^2$
\end{theorem}
\end{frame}

%------------------------------------------------

\begin{frame}[fragile] % Need to use the fragile option when verbatim is used in the slide
\frametitle{Verbatim}
\begin{example}[Theorem Slide Code]
\begin{verbatim}
\begin{frame}
\frametitle{Theorem}
\begin{theorem}[Mass--energy equivalence]
$E = mc^2$
\end{theorem}
\end{frame}\end{verbatim}
\end{example}
\end{frame}

%------------------------------------------------

\begin{frame}
\frametitle{Figure}
Uncomment the code on this slide to include your own image from the same directory as the template .TeX file.
%\begin{figure}
%\includegraphics[width=0.8\linewidth]{test}
%\end{figure}
\end{frame}

%------------------------------------------------

\begin{frame}[fragile] % Need to use the fragile option when verbatim is used in the slide
\frametitle{Citation}
An example of the \verb|\cite| command to cite within the presentation:\\~

This statement requires citation \cite{p1}.
\end{frame}

%------------------------------------------------

\begin{frame}
\frametitle{References}
\footnotesize{
\begin{thebibliography}{99} % Beamer does not support BibTeX so references must be inserted manually as below
\bibitem[Smith, 2012]{p1} John Smith (2012)
\newblock Title of the publication
\newblock \emph{Journal Name} 12(3), 45 -- 678.
\end{thebibliography}
}
\end{frame}

%------------------------------------------------

\begin{frame}
\huge{\centerline{Thank you for coming to my TED talk}}
\end{frame}

%----------------------------------------------------------------------------------------

\end{document} 